%---------- Inleiding ---------------------------------------------------------

% TODO: Is dit voorstel gebaseerd op een paper van Research Methods die je
% vorig jaar hebt ingediend? Heb je daarbij eventueel samengewerkt met een
% andere student?
% Zo ja, haal dan de tekst hieronder uit commentaar en pas aan.

%\paragraph{Opmerking}

% Dit voorstel is gebaseerd op het onderzoeksvoorstel dat werd geschreven in het
% kader van het vak Research Methods dat ik (vorig/dit) academiejaar heb
% uitgewerkt (met medesturent VOORNAAM NAAM als mede-auteur).
% 

\section{Inleiding}%
\label{sec:inleiding}

%Waarover zal je bachelorproef gaan? Introduceer het thema en zorg dat volgende zaken zeker duidelijk aanwezig zijn:

%\begin{itemize}
%  \item kaderen thema
%  \item de doelgroep
%  \item de probleemstelling en (centrale) onderzoeksvraag
%  \item de onderzoeksdoelstelling
%\end{itemize}

%Denk er aan: een typische bachelorproef is \textit{toegepast onderzoek}, wat betekent dat je start vanuit een concrete probleemsituatie in bedrijfscontext, een \textbf{casus}. Het is belangrijk om je onderwerp goed af te bakenen: je gaat voor die \textit{ene specifieke probleemsituatie} op zoek naar een goede oplossing, op basis van de huidige kennis in het vakgebied.

%De doelgroep moet ook concreet en duidelijk zijn, dus geen algemene of vaag gedefinieerde groepen zoals \emph{bedrijven}, \emph{developers}, \emph{Vlamingen}, enz. Je richt je in elk geval op it-professionals, een bachelorproef is geen populariserende tekst. Eén specifiek bedrijf (die te maken hebben met een concrete probleemsituatie) is dus beter dan \emph{bedrijven} in het algemeen.

%Formuleer duidelijk de onderzoeksvraag! De begeleiders lezen nog steeds te veel voorstellen waarin we geen onderzoeksvraag terugvinden.

%Schrijf ook iets over de doelstelling. Wat zie je als het concrete eindresultaat van je onderzoek, naast de uitgeschreven scriptie? Is het een proof-of-concept, een rapport met aanbevelingen, \ldots Met welk eindresultaat kan je je bachelorproef als een succes beschouwen?

De afgelopen maanden heb ik zowel stage gelopen als ook gewerkt als jobstudent bij het lokale gemeentebestuur in Berlare. Ook in een lokaal bestuur zitten we aan enorme toenamen aan digitalisering. De overheidsinstanties ervaren steeds een hogere werkdruk aangezien het takenpakket van de gemeenten steeds groter wordt. Dit zorgt er mede voor dat gemeenten vele administratieve processen hebben, waaronder onboarding er ook één van is. Verschillende diensten moeten dus samen werken om de nieuwe medewerkers in de verschillende systemen op te nemen. 

De doelgroep is dus de medewerkers van het lokale gemeentebestuur in Berlare. Zij zouden graag een verbetering zien in het onboardingproces zodat er minder fouten en tijdsverlies is. Niet enkel zijn de werknemers van de gemeente de doelgroep. Je zou ook kunnen zeggen dat de inwoners van de gemeente ook een deel zijn van de doelgroep aangezien een vlot werkend gemeentebestuur, voor hun ook voordelig is. 

Uit mijn stage en studentenjob werd mij duidelijk dat er bij de onboarding, wel eens menselijke fouten konden gemaakt worden. Verder waren er niet enkel menselijke fouten die ontstonden, er waren ook frustraties van collega's omdat ze moesten wachten tot er andere collega's hun deel van de onboarding deden. Bij problemen die ontstonden kan je denken aan collega's die de toegangscontrole voor de gebouwen niet in orde brachten, of de badge-nummer die niet werd doorgegeven zodat deze in de printserver opgeslagen konden worden. Deze probleemsituatie vormt de basis voor de centrale onderzoeksvraag: Op welke manier kan een geautomatiseerd systeem ontwikkeld worden dat, op basis van gegevens uit de personeelsdienst, een volledige en correcte onboarding kan voorzien binnen de gemeentelijke IT-infrastructuur? Deze vraag vertrekt vanuit de vaststelling dat het huidige proces tijdrovend en foutgevoelig is, en dat automatisatie potentieel kan bijdragen aan een efficiëntere werking.

De doelstelling is om een werkend prototype te hebben waarbij een medewerker van de personeelsdienst het onboardings proces start en alles daarna automatisch gebeurd. Hieruit zal dan ook de vergelijking kunnen gemaakt worden naar de huidige situatie en tijdsplanning ten opzichte van de uitgewerkte oplossing.



%---------- Stand van zaken ---------------------------------------------------

\section{Literatuurstudie}%
\label{sec:literatuurstudie}

%Hier beschrijf je de \emph{state-of-the-art} rondom je gekozen onderzoeksdomein, d.w.z.\ een inleidende, doorlopende tekst over het onderzoeksdomein van je bachelorproef. Je steunt daarbij heel sterk op de professionele \emph{vakliteratuur}, en niet zozeer op populariserende teksten voor een breed publiek. Wat is de huidige stand van zaken in dit domein, en wat zijn nog eventuele open vragen (die misschien de aanleiding waren tot je onderzoeksvraag!)?

%Je mag de titel van deze sectie ook aanpassen (literatuurstudie, stand van zaken, enz.). Zijn er al gelijkaardige onderzoeken gevoerd? Wat concluderen ze? Wat is het verschil met jouw onderzoek?

%Verwijs bij elke introductie van een term of bewering over het domein naar de vakliteratuur, bijvoorbeeld~\autocite{Hykes2013}! Denk zeker goed na welke werken je refereert en waarom.

%Draag zorg voor correcte literatuurverwijzingen! Een bronvermelding hoort thuis \emph{binnen} de zin waar je je op die bron baseert, dus niet er buiten! Maak meteen een verwijzing als je gebruik maakt van een bron. Doe dit dus \emph{niet} aan het einde van een lange paragraaf. Baseer nooit teveel aansluitende tekst op eenzelfde bron.

%Als je informatie over bronnen verzamelt in JabRef, zorg er dan voor dat alle nodige info aanwezig is om de bron terug te vinden (zoals uitvoerig besproken in de lessen Research Methods).

% Voor literatuurverwijzingen zijn er twee belangrijke commando's:
% \autocite{KEY} => (Auteur, jaartal) Gebruik dit als de naam van de auteur
%   geen onderdeel is van de zin.
% \textcite{KEY} => Auteur (jaartal)  Gebruik dit als de auteursnaam wel een
%   functie heeft in de zin (bv. ``Uit onderzoek door Doll & Hill (1954) bleek
%   ...'')

%Je mag deze sectie nog verder onderverdelen in subsecties als dit de structuur van de tekst kan verduidelijken.

Automatische gebruikers provisioning vormt vandaag een essentieel onderdeel van moderne toegangscontroles binnen organisaties. Denk bijvoorbeeld aan het toekennen van rechten op mappen maar ook aan de toegang tot gebouwen die openen met een badge. De nood aan geautomatiseerde onboarding processen is de afgelopen jaren sterk toegenomen door de groeiende complexiteit van IT-omgevingen en de afhankelijkheid van digitale middelen tijdens de eerste werkdag van een nieuwe medewerker. Volgens \textcite{Okta2025} leidt handmatige provisioning niet alleen tot inefficiëntie, maar vormt het ook een aanzienlijk beveiligingsrisico. Dit komt vooral omdat inconsistenties in het aanmaken van accounts en toewijzen van toegangsrechten het gevolg zijn van menselijke fouten. Automatische provisioning wordt in de vakliteratuur dan ook beschouwd als een noodzakelijke stap richting veilig identiteitsbeheer.

Uit onderzoek naar onboardingprocessen blijkt dat veel organisaties nog steeds vertrouwen op manuele workflows waarbij de personeelsdiensten, diensthoofden en IT elk afzonderlijke taken uitvoeren. \textcite{Radev2023} toont aan dat deze processen leiden tot vertragingen en een verhoogde administratieve last, waarbij fouten in gegevensoverdracht of ontbrekende stappen af en toe voorkomen. De studie benadrukt dat vooral het ontbreken van geïntegreerde systemen zorgt voor knelpunten, wat aansluit bij het probleem dat zich binnen de gemeente Berlare voordoet. \textcite{Vitla2023} bevestigt deze bevindingen en concludeert dat geautomatiseerde functionaliteit significant bijdraagt aan productiviteitswinst doordat medewerkers vanaf de eerste werkdag toegang hebben tot alle noodzakelijke digitale middelen. Dit bevordert zowel de eerste werkdag van de nieuwe collega, als ook de werktaken van de diensten die een handmatige taak hebben bij het onboardingsproces. 

Idem aan de voorgaande tekst blijft ook identiteitsmanagement zich sterk profileren als AI-ondersteunde decision making. \textcite{Hariharan2025} beschrijft hoe AI-gedreven IAM-systemen (identiteit- en toegangsmanagement systemen) risico's kunnen inschatten en automatisch de juiste rechten kunnen toekennen aan de hand van job descriptions en het detecteren van afwijkingen. Dergelijke oplossingen vinden hun toepassing voorlopig vooral bij grote bedrijven maar zijn in lokale besturen niet altijd toegestaan wegens de extra regels omdat er met persoonsgerelateerde gegevens wordt gewerkt. De technologie die in deze bachelorproef aan bod komt richt zich echter op een eenvoudiger schaalniveau waarbij automatisatie vooral gaat via API's, workflow-systemen of tools zoals Microsoft Power Automate of andere pijplijn systemen. Deze systemen beantwoorden dan ook de vraag over welke bestaande systemen er mogelijk opgenomen zouden moeten worden in de oplossing? Als ook op de vraag welke technologieën hiervoor beschikbaar zijn? 

Wat opvalt is dat de bestaande literatuur grotendeels betrekking heeft op enterprise-omgevingen en grote IAM-platformen. Studies die specifiek naar kleinere overheidsinstanties of gemeentelijke diensten kijken zijn zeldzaam. Veel onderzoeken beschrijven algemene onboarding processen, maar gaan niet in op de problemen die ontstaan bij het synchroniseren van systemen van een personeelsdienst met Active Directory in publieke organisaties. Er is duidelijk nog een gebrek aan kennis die aanleiding is tot dit onderzoek. Terwijl veel literatuur zich richt op modellen en implementaties van grote bedrijfsomgevingen, wil deze bachelorproef een stappenplan ontwikkelen dat als proof of concept kan dienen voor een gemeentelijke IT-afdeling. Hierbij wordt rekening gehouden met lokale werkwijzen, de bestaande infrastructuur en wettelijke vereisten zoals de GDPR.

Samengevat laat de huidige situatie duidelijk zien dat automatische gebruikers provisioning wordt gezien als een must voor een efficiënt en veilig identiteitsbeheer. Literatuur beschrijft wat de voordelen zijn, welke inefficiënties er in bestaande processen zitten en geeft aan hoe je technisch kunt integreren. Echter er wordt minder aandacht besteed aan de implementatie in kleinere gemeenten. Deze bachelorproef wil juist dat gat opvullen door te onderzoeken hoe een geautomatiseerd provisioning proces op basis van personeelsgegevens in een gemeentelijke ICT-omgeving kan worden ​‍​‌‍​‍‌gerealiseerd.

%---------- Methodologie ------------------------------------------------------
\section{Methodologie}%
\label{sec:methodologie}

%Hier beschrijf je hoe je van plan bent het onderzoek te voeren. Welke onderzoekstechniek ga je toepassen om elk van je onderzoeksvragen te beantwoorden? Gebruik je hiervoor literatuurstudie, interviews met belanghebbenden (bv.~voor requirements-analyse), experimenten, simulaties, vergelijkende studie, risico-analyse, PoC, \ldots?

%Valt je onderwerp onder één van de typische soorten bachelorproeven die besproken zijn in de lessen Research Methods (bv.\ vergelijkende studie of risico-analyse)? Zorg er dan ook voor dat we duidelijk de verschillende stappen terug vinden die we verwachten in dit soort onderzoek!

%Vermijd onderzoekstechnieken die geen objectieve, meetbare resultaten kunnen opleveren. Enquêtes, bijvoorbeeld, zijn voor een bachelorproef informatica meestal \textbf{niet geschikt}. De antwoorden zijn eerder meningen dan feiten en in de praktijk blijkt het ook bijzonder moeilijk om voldoende respondenten te vinden. Studenten die een enquête willen voeren, hebben meestal ook geen goede definitie van de populatie, waardoor ook niet kan aangetoond worden dat eventuele resultaten representatief zijn.

%Uit dit onderdeel moet duidelijk naar voor komen dat je bachelorproef ook technisch voldoen\-de diepgang zal bevatten. Het zou niet kloppen als een bachelorproef informatica ook door bv.\ een student marketing zou kunnen uitgevoerd worden.

%Je beschrijft ook al welke tools (hardware, software, diensten, \ldots) je denkt hiervoor te gebruiken of te ontwikkelen.

%Probeer ook een tijdschatting te maken. Hoe lang zal je met elke fase van je onderzoek bezig zijn en wat zijn de concrete \emph{deliverables} in elke fase?

In de eerste fase van het onderzoek zal er een studie van het concrete systeem bij de gemeente zijn. Dit moet uiteraard zijn omdat ik een antwoord moet hebben op de vraag welke stappen er momenteel al gebruikt worden voor dit onboarding proces als ook hoeveel tijd dit in beslag neemt? Hierbij zal ik bij de medewerkers polsen in welke mate zij taken moeten uitvoeren en welke systemen zij hiervoor gebruiken. Dit zorgt er dus voor dat ik gerichter naar een oplossing zal kunnen zoeken die toepasselijk is voor de gemeente Berlare. Voor naar een goede oplossing te gaan, zal ik ook antwoorden nodig hebben op de vraag welke fouten er het meeste voorkomen? Hieruit zal dan een lijst met requirements komen. Deze zullen gesorteerd moeten worden op vlak van functionaliteit en belang. Op vlak van timing, zou dit ongeveer in de tweede week van februari afgerond zijn. 

Nadat er een idee is van welke systemen nodig zijn en gebruikt worden, zal er een literatuurstudie gedaan worden. Deze studie zal ervoor zorgen dat ik weet op welke manieren er nu al van deze geautomatiseerde systemen beschikbaar zijn voor het gebruik. Dit onderzoek zal er dan ook voor zorgen dat ik weet richting welke systemen ik zal trekken om mijn proof of concept uit te werken. Tijdens het literatuuronderzoek zal ik dan ook een lijst moeten opmaken met welke systemen daadwerkelijk een mogelijke positieve invloed op het onboardings proces zouden hebben. Tegen eind februari zou er voldoende literatuur gevonden moeten zijn.

Daarna begint de laatste en grootste fase. Het uitwerken van een proof of concept. Hierbij zal ik de informatie uit de vorige 2 fases gebruiken om een zo toepasselijk mogelijke opstelling te maken. Dit zodat het uiteindelijke doel bereikt kan worden. Momenteel vermoed ik dat er API-koppelingen gebruikt zullen worden die vanuit Microsoft Planner gelinkt zullen worden aan verschillende attributen in de active directory van de gemeente. Mogelijk zouden pijplijnen ook kunnen helpen bij het robuust maken van deze systemen. Denk hieraan aan systemen zoals Jenkins of Power Automate. 

Hoewel de fasen chronologisch worden voorgesteld, worden ze iteratief doorlopen: inzichten uit de PoC-fase kunnen aanleiding geven tot bijkomende literatuuronderzoek of extra vragen aan medewerkers. Dit zorgt ervoor dat het eindresultaat realistisch, toepasbaar en technisch onderbouwd is.




%---------- Verwachte resultaten ----------------------------------------------
\section{Verwacht resultaat, conclusie}%
\label{sec:verwachte_resultaten}

%Hier beschrijf je welke resultaten je verwacht. Als je metingen en simulaties uitvoert, kan je hier al mock-ups maken van de grafieken samen met de verwachte conclusies. Benoem zeker al je assen en de onderdelen van de grafiek die je gaat gebruiken. Dit zorgt ervoor dat je concreet weet welk soort data je moet verzamelen en hoe je die moet meten.

%Wat heeft de doelgroep van je onderzoek aan het resultaat? Op welke manier zorgt jouw bachelorproef voor een meerwaarde?

%Hier beschrijf je wat je verwacht uit je onderzoek, met de motivatie waarom. Het is \textbf{niet} erg indien uit je onderzoek andere resultaten en conclusies vloeien dan dat je hier beschrijft: het is dan juist interessant om te onderzoeken waarom jouw hypothesen niet overeenkomen met de resultaten.

Op basis van de eerdere analyse en de doelstellingen van het onderzoek wordt verwacht dat de ontwikkeling van een geautomatiseerd onboardingproces binnen het gemeentebestuur van Berlare zal leiden tot een duidelijke verbetering in efficiëntie, foutvermindering en consistentere uitkomst van het proces. De centrale hypothese hierbij is dat een systeem waarbij de personeelsdienst slechts éénmaal gegevens invoert, de manuele werklast voor de IT-dienst en andere diensten in het onboardingproces aanzienlijk zal verminderen en de betrouwbaarheid van het volledige onboardingproces zal verhogen.

Tijdens de evaluatie van de proof-of-concept zal onder meer de doorlooptijd van het onboardingproces gemeten worden. Hierbij wordt verwacht dat deze merkbaar korter zal zijn dan in de huidige situatie, omdat veel handmatige stappen en afhankelijkheden tussen verschillende diensten wegvallen. Daarnaast wordt aangenomen dat het aantal fouten binnen het proces afneemt, met name bij configuraties die vandaag gevoelig zijn voor manuele vergissingen, zoals toegangsrechten, accountinstellingen en printer- of badge configuraties. Door het gebruik van gestandaardiseerde automatisatiescripts en API-koppelingen zal de foutmarge vermoedelijk veel lager liggen dan in het huidige, manuele proces.

De meerwaarde voor de doelgroep — de personeelsdienst en de IT-dienst van het gemeentebestuur van Berlare — ligt vooral in de vermindering van werkdruk, het wegvallen van repetitieve taken en een grotere betrouwbaarheid van het onboardingproces. Indirect profiteert ook de organisatie als geheel, omdat een efficiënter onboardingproces de continuïteit van de dienstverlening aan inwoners ondersteunt.

Waaraan moet er voldaan worden op het einde van het project? Samengevat wordt verwacht dat het onderzoek zal aantonen dat een geautomatiseerd onboardingproces haalbaar en zinvol is binnen de bestaande IT-infrastructuur en dat het zowel kwantitatieve als kwalitatieve voordelen oplevert. Mocht blijken dat bepaalde onderdelen moeilijker te automatiseren zijn dan voorzien, dan levert dit alsnog waardevolle inzichten op in de technische of organisatorische beperkingen die verdere optimalisatie in de weg staan.